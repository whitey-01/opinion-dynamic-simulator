\documentclass[a4paper, 12pt]{report}
\usepackage[latin1]{inputenc}
\usepackage[italian]{babel}
%\usepackage[T1]{fontenc}
\usepackage{graphicx}
\usepackage{float}
\usepackage[centertags]{amsmath}
\usepackage{amsfonts}
\usepackage{amssymb}
\usepackage{amsthm}
\usepackage{newlfont}
\usepackage{fancyhdr}
\usepackage{tesisty}
\usepackage{subfiles}

%-------------------------------
% DEFINIZIONE DEGLI ENVIRONMENT
%-------------------------------

\newtheorem{obs}{Osservazione}[section]
\newenvironment{oss}
    {\begin{obs}\begin{normalfont}}
    {\hfill $\square \!\!\!\!\checkmark$ \end{normalfont}\end{obs}}

\newtheorem{pro}{Problema}[chapter]
\newenvironment{prob}
    {\begin{pro}\begin{normalfont}}
    {\hfill $\spadesuit$ \end{normalfont}\end{pro}}

\newtheorem{teor}{Teorema}[section]
\newenvironment{teorema}
    {\begin{teor}\textit }
    {\hfill  \end{teor}}

\newtheorem{defn}{Definizione}[section]
\newenvironment{de}
    {\begin{defn}\begin{normalfont}}
    {\hfill $\clubsuit$ \end{normalfont}\end{defn}}

%-----------------------------
% CONFIGURAZIONE DELLA PAGINA
%-----------------------------

\hfuzz2pt % Don't bother to report over-full boxes if over-edge is < 2pt

\fancypagestyle{plain}{
\fancyhead{}\renewcommand{\headrulewidth}{0pt} } \pagestyle{fancy}
\renewcommand{\chaptermark}[1]{\markboth{\small CAP. \thechapter \textit{ #1}} {} }
\renewcommand{\sectionmark}[1]{\markright{\small  \thesection \textit{ #1}} {} }
\voffset=-20pt    % distanza tra il limite superiore del foglio e l'intestazione
\headsep=40pt     % distanza  l'intestazione ed il testo del corpo
\hoffset=0 pt     % misura equivalente al margine sinistro
\textheight=620pt % altezza del corpo del testo
\textwidth=435pt  % larghezza del corpo del testo
\footskip=40pt    % distanza tra il testo del corpo ed il pie' di pagina
\fancyhead{}      % cancella qualsiasi impostazione per l'intestazione
\fancyfoot{}      % cancella qualsiasi impostazione per il pie' di pagina
\headwidth=435pt  % larghezza del'intestazione e del pie' di pagina
\fancyhead[R]{\rightmark} \fancyfoot[L]{\leftmark}
\fancyfoot[R]{\thepage}
\renewcommand{\headrulewidth}{0.3pt}   % spessore della linea dell'intestazione
\renewcommand{\footrulewidth}{0.3pt}   % spessore della linea del pi�di pagina

\numberwithin{equation}{section}
\renewcommand{\theequation}{\thesection.\arabic{equation}}




%--------------------------
% MODIFICARE DA QUI IN POI
%--------------------------

\begin{document}


\corso{INFORMATICA} \titoloTesi{Analisi dell'impatto di "bias" e topologia sui tempi di assorbimento per Dinamiche di Opinioni} \anno{2019/2020}
\relatore{Francesco Pasquale}
 \autore{Saverio Biancone}
\correlatore{Sara Rizzo}

\baselineskip=25pt

\intestazione

%------------------------------------------------
% INTRODUZIONE E RINGRAZIAMENTI (NON MODIFICARE)
%------------------------------------------------

\fancypagestyle{plain}{
\fancyhead{}\renewcommand{\headrulewidth}{0pt} } \pagestyle{fancy}
\renewcommand{\chaptermark}[1]{\markboth{\small Cap. \thechapter \textit{ #1}} {} }
\renewcommand{\sectionmark}[1]{\markright{\small  \S \thesection \textit{ #1}} {} }
\voffset=-20pt                         % distanza tra il limite superiore del foglio e l'intestazione
\headsep=40pt                          % distanza  l'intestazione ed il testo del corpo
\hoffset=0pt                           % misura equivalente al margine sinistro
\textheight=620pt                      % altezza del corpo del testo
\textwidth=435pt                       % larghezza del corpo del testo
\footskip=40pt                         % distanza tra il testo del corpo ed il pie' di pagina
\fancyhead{}                           % cancella qualsiasi impostazione per l'intestazione
\fancyfoot{}                           % cancella qualsiasi impostazione per il pie' di pagina
\headwidth=435pt                       % larghezza del'intestazione e del pie' di pagina
\fancyhead[R]{\rightmark} \fancyfoot[L]{\leftmark}
\fancyfoot[R]{\thepage}
\renewcommand{\headrulewidth}{0.3pt}   % spessore della linea dell'intestazione
\renewcommand{\footrulewidth}{0.3pt}   % spessore della linea del pi�di pagina

\pagenumbering{Roman} \tableofcontents
\newpage

\pagenumbering{arabic}

\fancyhead[R]{Introduzione} \fancyfoot[L]{Introduzione}
\fancyfoot[R]{\thepage}

\include{Introduzione}

\fancyhf{} %elimina header/footer vecchi


\fancyhead[R]{\rightmark} \fancyhead[L]{\leftmark}
\fancyfoot[R]{\thepage}





%---------------------
% INCLUSIONE CAPITOLI
%---------------------

\subfile{capitoli/capitolo1}
\subfile{capitoli/capitolo2}
\subfile{capitoli/capitolo3}
\subfile{capitoli/capitolo4}




% ELENCO DELLE FIGURE (OPZIONALE)
\addcontentsline{toc}{chapter}{Elenco delle figure}
\listoffigures


% BIBLIOGRAFIA
\addcontentsline{toc}{chapter}{Bibliografia}
\begin{thebibliography}{9}
        
\end{thebibliography}
\end{document}
