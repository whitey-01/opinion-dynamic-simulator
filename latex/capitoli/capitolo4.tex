\documentclass[../Tesi.tex]{subfiles}

\begin{document}
\chapter{Strumenti di Analisi}
\section{Opinion Dynamic Simulator}
Il software sviluppato fornisce degli strumenti utili a riprodurre le dinamiche descritte al fine di indagare legami inespressi che intercorrono tra alcune topologie e valori di bias.\\*
Il linguaggio principale scelto per lo sviluppo del software \'e Python. Le motivazioni dietro tale scelta risiedono non soltanto nella grande diffusione e nel grande supporto di cui questo linguaggio gode, ma anche dalla presenza di numerosi moduli che permettono di rappresentare ed interagire al meglio con i grafi, strutture attraverso le quali viene rappresentata la rete sociale.
Tra i vari moduli disponibili \'e stato scelto Graph-Tool. Tra i punti di forza di questo modulo si annoverano una precisa modellazione di tali strutture, con possibilit\'a di aggiungere propriet\'a personalizzate su vertici e archi nonch\'e numerose funzioni di generazione e rappresentazione di alcune topologie tipiche dei grafi, quali Paths, Cicli e Cliques. Il vero punto di forza di Graph-Tool \'e per\'o la sua velocit\'a, ottenuta grazie alla possibilit\'a di eseguire molte di queste funzioni non in Python, bens\'i in C, linguaggio di livello pi\'u basso e perci\'o pi\'u rapido nell'esecuzione di alcune istruzioni fondamentali.\\*
Alcune topologie non presenti nativamente nel modulo, sono state personalmente riscritte utilizzando comunque le strutture di base fornite. \'E questo il caso dell'Ipercubo, per il quale ho scritto un algoritmo di generazione basato sull'idea alla base di Bit-Fix\footnote{Algoritmo utilizzato per il ``routing'' su ipercubi.}, e del Modello di Erd{\"o}s R\'enyi.\\*
Partendo da tale base, \'e stato sviluppato il processo di simulazione vero e proprio. Tale processo necessita come input, oltre al grafo, di un oggetto (d'ora in poi definito come \emph{configurator}) contenente i parametri necessari alla configurazione della simulazione, come ad esempio la dinamica adottata e il bias verso l'opinione dominante.\\* 
Al termine del processo di simulazione verranno salvate diverse informazioni, quali:
\begin{itemize}
\item Un file .XML contenente la serializzazione del grafo, cosi che possa essere facilmente ricostruito e riutilizzato.
\item Un file .XML contenente le informazioni di configurazione e il risultato (numero di step) della simulazione.
\item Una rappresentazione grafica del grafo in formato .PNG
\item Una cartella contenente dei file .PNG che mostrano l'andamento evolutivo della simulazione.
\end{itemize}
Al fine di ottenere dei dati quanto pi\'u possibile affidabili e limitare l'aleatoriet\'a di tali processi, il software dispone inoltre di un modulo volto all'esecuzione di test. In questo contesto per test si intende l'esecuzione ripetuta di una simulazione, mantenendo invariati i parametri di configurazione e la struttura del grafo. Cos\'i come le simulazioni, anche i test necessitano di una configurazione che include, oltre al numero di iterazioni da effettuare, l'oggetto configurator riferito alle simulazioni eseguite nel test.
Ponendo come obiettivo quello di velocizzare l'esecuzione di un test e ridurre la dimensione dell'output, soprattutto per grafi con un numero elevato di vertici e archi, l'esecuzione di un test non comporta il salvataggio di tutte le informazioni relative alle singole simulazioni, ma solamente:
\begin{itemize}
\item Un file .XML contenente la serializzazione del grafo, cosi che possa essere facilmente ricostruito e riutilizzato.
\item Un file .XML contenente le informazioni di configurazione del test, i risultati di ogni singola simulazione, la media di tali valori e la deviazione standard.
\item Una rappresentazione grafica del grafo in formato .PNG.
\end{itemize}
Considerando la durata media di un test composto da 100 simulazioni, eseguito su un grafo particolarmente impegnativo, ad esempio con $2^{12^{\mathrm{}}}$ vertici, ho scelto di inserire un modulo che permettesse di riunificare molteplici test pre-eseguiti, utilizzando come dataset l'insieme delle simulazioni eseguite e ricalcolando correttamente la media e la deviazione standard.\\*
In questo modo \'e possibile suddividere un test composto da 100 simulazioni in 10 test indipendenti, da 10 simulazioni ognuno, permettendo cos\'i di ridurre sensibilmente lo stress computazionale inflitto alla macchina sul quale viene eseguito.\\
La scelta del formato .XML \'e dettata dalla necessit\'a di avere quante pi\'u informazioni possibili salvate in maniera strutturata e coerente, grazie anche alla possibilit\'a, fornita nativamente da Python, di formattare automaticamente files in tale estensione.\\
Un ulteriore vantaggio dato dalla scelta del formato .XML \'e la possibilit\'a di re-importare un grafo (tramite un metodo fornito dalla libreria graph-tool), partendo dal file ottenuto dalla serializzazione dello stesso. Questo risulta molto efficiente qualora si volessero eseguire ulteriori test/simulazioni a partire da un grafo precedentemente generato e difficilmente ricostruibile (nel caso di topologie ottenute tramite processi aleatori come nel Modello di Erd{\"o}s R\'enyi).
Le rappresentazioni grafiche invece sono state salvate in .PNG, un buon compromesso tra qualit\'a dell'output e dimensione dei file stessi. Altre opzioni facilmente implementabili sono .SVG e .PDF.\\*
Il \emph{core} del software \'e rappresentato dall'algoritmo di simulazione che esegue il processo su un grafo fornito in input, aggiornando l'opinione di ogni nodo attraverso una \emph{vertex property} definita appositamente.\\*
L'algoritmo di simulazione supporta attualmente \emph{Voter Model} e \emph{Majority-Dynamics} ma il codice \'e stato modellato in modo da poter facilmente implementare, mediante un'interfaccia\footnote{Python non fornisce nativamente le interfacce, perci\'o \'e stata simulata utilizzando classi e metodi astratti.}, ulteriori dinamiche di aggiornamento.
\end{document}