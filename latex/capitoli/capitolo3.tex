\documentclass[../Tesi.tex]{subfiles}

\begin{document}
\chapter{Studi svolti e Analisi dei risultati}
Basandomi sui risultati descritti dal paper [1] e riportati nel precedente capitolo, ho tentato di rispondere ad alcune delle domande rimaste insolute.\\*
La mia attenzione \'e stata rivolta in particolare al tempo di assorbimento che si ottiene adottando Majority-Dynamic su alcune topologie quando il bias \'e inferiore a $\frac{1}{2}$.\\*
Per poter comparare ed analizzare statisticamente i tempi di assorbimento, ho sviluppato un software in grado di simulare il modello preso in esame sulle seguenti topologie:
 \begin{itemize}
\item Ipercubo
\item Clique
\item Ciclo
\item Modello di Erd{\"o}s R\'enyi G($n$,$p$)
\end{itemize}
\section{Analisi di Clique, Ipercubo e Ciclo}
I primi test eseguiti sono serviti a verificare l'accuratezza del software e correggerne eventuali errori. Ogni test \'e composto da 100 simulazioni, configurate con Majority-Dynamic e bias pari a $\frac{1}{2}$.\\*
I test sono stati eseguiti su Ipercubi, Cliques e Cicli, in dimensioni che vanno da $2^{5^{\mathrm{}}}$ fino a $2^{12^{\mathrm{}}}$ vertici.\\*
I risultati ottenuti (fig. 3.1.1) sono perfettamente in linea con quanto dimostrato teoricamente in  "Biased Opinion Dynamics" [1]. Per quanto riguarda il Ciclo, il tempo di assorbimento \'e stato pari a $O(\frac{1}{\alpha}n\log{}n)$. \\*
Riguardo Clique e Ipercubo, per i quali il grado minimo \'e pari a $\Omega(\log{}n)$, il tempo di assorbimento con bias pari a $\frac{1}{2}$ si \'e rivelato essere $O(n\log{}n)$, anche questa volta confermando quanto evidenziato dal paper [1].\\*
Una volta appurato perci\'o che il software simulasse correttamente i processi descritti, ottenendo risultati conformi a quanto atteso, sono stati eseguiti test (fig. 3.1.2) metodologicamente analoghi ai precedenti, adottando per\'o un bias verso l'opinione dominante pari a $\frac{1}{4}$, di fatto dimezzandolo.\\*
Per quanto riguarda il Ciclo e la Clique, sono stati ancora una volta confermati i risultati attesi.
Lo stato di assorbimento per il Ciclo \'e stato nuovamente raggiunto in $O(\frac{1}{\alpha}n\log{}n)$ passi in valore atteso. Nello specifico, dimezzando il valore del bias, i tempi di assorbimento per tale topologia sono raddoppiati, suggerendo una proporzionalit\'a inversa tra il tempo di assorbimento e il valore del bias.\\*
D'altra parte, come previsto, \'e stato impossibile ultimare i test per la Clique, in quanto, con un bias inferiore a $\frac{1}{2}$, il numero di passi necessari a raggiungere lo stato di assorbimento \'e diventato esponenziale nel grado minimo ($n-1$ per le Clique), di fatto rendendo impossibile lo svolgimento dei test gi\'a dalla dimensione di $2^{6^{\mathrm{}}}$.\\*
Il paper in esame [1], d'altronde, lascia aperta la domanda inerente al comportamento dell'Ipercubo con un bias cos\'i basso. Dai risultati dei test si evince come l'Ipercubo abbia un comportamento assimilabile a quello del Ciclo, raggiungendo perci\'o lo stato di assorbimento in $O(n\log{}n)$ passi.
\section{Analisi del Modello di Erd{\"o}s R\'enyi}
Questa sezione \'e dedicata all'analisi dei tempi di assorbimento per topologie non regolari, cio\'e grafi per i quali i vertici non condividono lo stesso grado.\\
Tali topologie sono state generate scrivendo un algoritmo aleatorio basato sul Modello di Erd{\"o}s R\'enyi G($n$,$p$).\\*
In tale modello, dato un grafo G($V$,$E$) con $|V|$ = $n$,
\begin{equation}
    \forall \: u,v \in V,\; (u,v) \in E  \text{ con probabilit\'a } p.
\end{equation}
I test eseguiti sono composti da 100 simulazioni, configurate con Majority-Dynamic e bias pari a $\frac{1}{4}$.\\*
Le topologie prese in considerazione hanno dimensioni che vanno da $2^{5^{\mathrm{}}}$ fino a $2^{12^{\mathrm{}}}$ vertici e sono state generate con il modello G($n$,$p$) descritto precedentemente, utilizzando i seguenti valori di $p$:
\begin{itemize}
\item $p$ $\leq$ $\frac{1-\epsilon}{n}$, per il quale il grafo risulta principalmente sparso, con alta probabilit\'a
\item $p$ $\geq$ $\frac{1+\epsilon}{n}$, per il quale il grafo presenta una $giant$ $component$ circondata da componenti pi\'u piccole di dimensione $O(\log{}n)$, con alta probabilit\'a
\item $p$ = $\frac{\log{}n}{n}$, per il quale il grafo risulta connesso e discretamente denso, con alta probabilit\'a
\end{itemize}
Esempi: $n$ = 256, $\epsilon$ = $\frac{1}{2}$
I risultati dei test descrivono un andamento meno prevedibile di quanto ci si potesse aspettare.\\*
Per valori di p che si allontanano dallo 0, il tempo di assorbimento inizia a diminuire, mentre gli studi compiuti sulla Clique [1] evidenziano una crescita esponenziale nel grado minimo quando $p$ tende a 1. La curva che si ottiene perci\'o fissando $n$ e facendo variare $p$ non risulta essere monotona crescente.\\*
Per indagare tale comportamento in modo pi\'u approfondito, ho eseguito ulteriori test con la seguente configurazione:
\begin{itemize}
    \item $n=512$ (fig. 3.2.2) e $n=1024$ (fig. 3.2.3)
    \item $\alpha=\frac{1}{4}$
    \item $p$ assume valori in [ $\frac{1-\epsilon}{n}$ , $\frac{{(1+\epsilon)}\log{}n}{n}$ ] t.c. $p_{i+1}$ - $p_{i}$ = 0.001
\end{itemize}\\*
Di seguito un grafico che illustra il tempo di assorbimento al variare di $p$.\\*
Sulle ascisse sono rappresentati i valori di $p$ mentre sulle ordinate i tempi di assorbimento. La curva in verde rappresenta il valore medio ricavato dai test, mentre le curve in blu e rosso descrivono i bound ottenuti rispettivamente sottraendo e sommando la deviazione standard al valore medio.
\end{document}