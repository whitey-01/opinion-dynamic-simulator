\documentclass[../Tesi.tex]{subfiles}

\begin{document}
\chapter{Introduzione}
Questo lavoro di tesi verte sullo studio e sull'analisi di un particolare modello di Dinamica delle Opinioni in cui in un sistema binario gli agenti esprimono una preferenza verso un'opinione dominante.\\*
Partendo dai risultati ottenuti dagli autori dell'articolo \emph{Biased Opinion Dynamics} \cite{DBLP:journals/corr/abs-2008-13589}, ho cercato di trovare risposte ad alcune delle domande rimaste insolute e proposte nelle conclusioni dell'articolo stesso. Questo studio tenta di individuare legami inespressi tra alcune topologie caratteristiche e la dinamica di aggiornamento \emph{Majority-Dynamics} quando il bias verso l'opinione dominante \'e inferiore a $\frac{1}{2}$.\\*
Per far questo \'e stato sviluppato un software in Python in grado di simulare i processi descritti cosi da poter effettuare un'analisi empirica attraverso i dati ricavati. Le topologie maggiormente trattate in questo lavoro sono state \emph{Ipercubo} e \emph{Modello di Erd{\"o}s R\'enyi G($n$,$p$)}\cite{Erdos:1959:pmd} seppure spesso sia stato utile poter confrontare i dati con quelli ottenuti da \emph{Clique} e \emph{Ciclo}, i quali comportamenti sono stati gi\'a ampiamente osservati nell'articolo \cite{DBLP:journals/corr/abs-2008-13589}. 
\end{document}